\documentclass[a4paper]{article}

%% Language and font encodings
\usepackage[english,polish]{babel}
\usepackage[utf8x]{inputenc}
\usepackage[T1]{fontenc}
\usepackage{subfloat}

%% Sets page size and margins.00
\usepackage[a4paper,top=3cm,bottom=2cm,left=3cm,right=3cm,marginparwidth=1.75cm]{geometry}

%% Useful packages
\usepackage{amsmath}
\usepackage{graphicx}
\usepackage[colorinlistoftodos]{todonotes}
\usepackage{float}
\usepackage{subfig}
\usepackage[colorlinks=true, allcolors=blue]{hyperref}

\begin{document}
	\begin{center}
		{\huge Sieci Komputerowe 2} \\
		{\LARGE Projekt komunikatora typu skype}\\
		Krzysztof Sychla (136807),
		Damian Wędzikowski (136826)
	\end{center}
	
	\section*{Temat}
	
	\section*{Działanie}
		Serwer po uruchomieniu rozpoczyna swoją pracę skonfigurowaniu gniazda sieciowego, po czym akceptuje przychodzące połączenia od klientów, a następnie tworzy wątek obsługi dla każdego z nich. Zadaniem tych wątków jest czytanie przychodzących ramek od klientów. Ramki te rozpoczynają się od nagłówków.
		\subsection*{Nagłówki analizowane po stronie serwera}
		\begin{itemize}
			\item N - przyjęcie wysłanego nicku i wpisanie go do mapy, współdzielonej między wszystkimi wątkami
			\item L - odesłanie klientowi listy wszystkich połąćzeń w systemie
			\item D - obsługa proźby o rozłączenie klienta
			\item C - obsługa prośby o połączenie z innym klientem, po czym przejście do funkcji zajmującej się przesyłaniem klatek między klientami
			\item A - 
		\end{itemize}
	
\end{document}